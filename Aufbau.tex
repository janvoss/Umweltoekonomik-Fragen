\documentclass[border=10pt]{standalone}
\usepackage{smartdiagram}
\usepackage[ngerman]{babel}
\usepackage[utf8]{inputenc} %UTF-Code für Umlaute
\usepackage[T1]{fontenc} %Trennung von Wörtern mit Umlauten
\usepackage{lmodern}
%\usepackage{libertine}
\renewcommand*\familydefault{\sfdefault} %Serifenlose Schrift als Standard
%\usepackage{microtype}
\usesmartdiagramlibrary{additions}
\usetikzlibrary{fit}

\tikzstyle{container} = [draw, rectangle, semithick, inner sep=0.3cm
]

%Aufzählungsstriche
\AtBeginDocument{
	\def\labelitemi{\normalfont\bfseries{--}}
}


\begin{document}

		\begin{tikzpicture}[
		every node/.style = {shape=rectangle, % is not necessary, default node's shape is rectangle
			rounded corners,
			draw, semithick,
			text width=5cm,
			align=left,
			%node distance=0.3cm
		}
		]
		\node (Einleitung-Erläuterung)[
      %  ,draw=none
      ]{
			\begin{itemize}
           \item Worum geht es?
           \item Warum ist es interessant?
          \item Was haben Sie gemacht? Was ist besonders daran?
            \item Wesentliche Ergebnisse (und warum sie interessant sind)
            \item Weiterer Aufbau 
			\end{itemize}
        };
		\node (Einleitung)[draw=none, left= of Einleitung-Erläuterung, align=left]{\textbf{Einleitung}\\ Hier holen Sie die  Zuhörer ab und binden sie sich.
		};
		
		\node (Hauptteil-Erläuterung)[
        below= of Einleitung-Erläuterung
       % ,draw=none
        ]{
			\begin{itemize}
            \item Thematische Grundlagen/ Stand der Literatur
            \item Ggf. methodische Grundlagen
            \item Sie berichten, was Sie gemacht haben
            \item Ergebninsse
            \item Ggf. Diskussion der Ergebnisse
		\end{itemize}
		};
		
		\node (Hauptteil)[draw=none, left= of Hauptteil-Erläuterung, align=left]{\textbf{Hauptteil}\\ Hier zeigen Sie, was Sie gemacht haben, dass es besonders ist und richten alles auf ihre Ergebnisse hin aus};
		
		\node (Fazit-Erläuterung)[
        below= of Hauptteil-Erläuterung
       % ,draw=none
       ]{	
			\begin{itemize}
			\item Wesentliche Ergebnisse
			\item Ggf. Warum ist Ihre Methode/ sind Ihre Ergebnisse besonders
			\item Einordnung
			\item Ausblick
			\end{itemize}
            };
		
		\node (Fazit)[draw=none, left= of Fazit-Erläuterung, align=left]{\textbf{Fazit}\\ Was sind die Kernbotschaften? Was soll eine zuhörende Person auf jeden Fall behalten?};
        

		\node (LVZ-Erläuterung)[
        below= of Fazit-Erläuterung,
       % draw=none
        ]{
        \begin{itemize}
        \item Nur \textbf{ein} Literaturverzeichnis
        \item alphabetisch sortiert
        \item vollständig
        \end{itemize}
        };  
        
        \node (LVZ)[draw=none, left= of LVZ-Erläuterung]{\textbf{Literaturverzeichnis}\\ Bei Konferenzvorträgen unüblich. Aber Sie machen es bitte};
		
		\node (Anhänge-Erläuterung)[
        below= of LVZ-Erläuterung,
       ]{
        Zusatzmaterial, das man auf Nachfrage hervorzaubern kann
        };
		
		\node (Anhänge)[draw=none, left= of Anhänge-Erläuterung]{\textbf{Backup Folien}};
		
	%	\node (Hauptcontainer) [container,fit=(Einleitung)(Hauptteil)(Fazit)]{};
		
	%	\node (Vorcontainer) [container,fit=(Deckblatt)(Verzeichnisse)]{};		
	%	\node (Nachcontainer) [container,fit=(Anhänge)]{};
		\end{tikzpicture}

\end{document}